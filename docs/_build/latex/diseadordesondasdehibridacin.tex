%% Generated by Sphinx.
\def\sphinxdocclass{report}
\documentclass[letterpaper,10pt,spanish]{sphinxmanual}
\ifdefined\pdfpxdimen
   \let\sphinxpxdimen\pdfpxdimen\else\newdimen\sphinxpxdimen
\fi \sphinxpxdimen=.75bp\relax
\ifdefined\pdfimageresolution
    \pdfimageresolution= \numexpr \dimexpr1in\relax/\sphinxpxdimen\relax
\fi
%% let collapsible pdf bookmarks panel have high depth per default
\PassOptionsToPackage{bookmarksdepth=5}{hyperref}

\PassOptionsToPackage{booktabs}{sphinx}
\PassOptionsToPackage{colorrows}{sphinx}

\PassOptionsToPackage{warn}{textcomp}
\usepackage[utf8]{inputenc}
\ifdefined\DeclareUnicodeCharacter
% support both utf8 and utf8x syntaxes
  \ifdefined\DeclareUnicodeCharacterAsOptional
    \def\sphinxDUC#1{\DeclareUnicodeCharacter{"#1}}
  \else
    \let\sphinxDUC\DeclareUnicodeCharacter
  \fi
  \sphinxDUC{00A0}{\nobreakspace}
  \sphinxDUC{2500}{\sphinxunichar{2500}}
  \sphinxDUC{2502}{\sphinxunichar{2502}}
  \sphinxDUC{2514}{\sphinxunichar{2514}}
  \sphinxDUC{251C}{\sphinxunichar{251C}}
  \sphinxDUC{2572}{\textbackslash}
\fi
\usepackage{cmap}
\usepackage[T1]{fontenc}
\usepackage{amsmath,amssymb,amstext}
\usepackage{babel}



\usepackage{tgtermes}
\usepackage{tgheros}
\renewcommand{\ttdefault}{txtt}



\usepackage[Sonny]{fncychap}
\ChNameVar{\Large\normalfont\sffamily}
\ChTitleVar{\Large\normalfont\sffamily}
\usepackage{sphinx}

\fvset{fontsize=auto}
\usepackage{geometry}


% Include hyperref last.
\usepackage{hyperref}
% Fix anchor placement for figures with captions.
\usepackage{hypcap}% it must be loaded after hyperref.
% Set up styles of URL: it should be placed after hyperref.
\urlstyle{same}

\addto\captionsspanish{\renewcommand{\contentsname}{Contents:}}

\usepackage{sphinxmessages}
\setcounter{tocdepth}{1}



\title{Diseñador de sondas de hibridación}
\date{27 de noviembre de 2023}
\release{}
\author{Simón Vergara}
\newcommand{\sphinxlogo}{\vbox{}}
\renewcommand{\releasename}{}
\makeindex
\begin{document}

\ifdefined\shorthandoff
  \ifnum\catcode`\=\string=\active\shorthandoff{=}\fi
  \ifnum\catcode`\"=\active\shorthandoff{"}\fi
\fi

\pagestyle{empty}
\sphinxmaketitle
\pagestyle{plain}
\sphinxtableofcontents
\pagestyle{normal}
\phantomsection\label{\detokenize{index::doc}}


\sphinxstepscope


\chapter{Probe\sphinxhyphen{}Designer}
\label{\detokenize{modules:probe-designer}}\label{\detokenize{modules::doc}}
\sphinxstepscope


\section{descarga module}
\label{\detokenize{descarga:descarga-module}}\label{\detokenize{descarga::doc}}
\sphinxAtStartPar
Para obtener el ADN que se utilizará como referencia para el diseño de las sondas se va a descargar directamente el archivo genético desde la base de datos ‘Nucleotide’ de NCBI (National Center for Biotechnology Information) del NIH (National Institute of Health, EEUU) (\sphinxurl{https://www.ncbi.nlm.nih.gov/nuccore}). Esta base de datos contiene millones de registros de secuencias de nucleótidos con mucha información adicional, como la especie, cromosoma, nombre del gen, exones, la fuente de los datos, etc.

\sphinxAtStartPar
Para llevar a cabo la descarga de los archivos genéticos, se implementó un módulo que permite realizar descargas desde esta base de datos a través de la librería Entrez de BioPython y almacenar los archivos en una carpeta. De esta manera se puede automatizar la obtención de las secuencias de referencia.

\sphinxAtStartPar
Por otra parte se implementó la función que retorna una secuencia a partir de un archivo almacenado localmente, además de crear una copia del archivo y almacenarla. Esta función sólo podrá recibir archivos de secuencia anotados como GenBank y GFF3.

\sphinxAtStartPar
El producto de estas funciones implementadas es la secuencia en formato SeqRecord, una estructura de datos propia de la librería BioPython que permite almacenar grandes secuencias con su respectiva notación, como por ejemplo las posiciones de cada uno de los exones, las transcripciones, secciones no codificantes, entre otros. Además contiene información extra de la secuencia tal como la especie, el cromosoma, la fuente de donde salieron los datos, etc.
\index{module@\spxentry{module}!descarga@\spxentry{descarga}}\index{descarga@\spxentry{descarga}!module@\spxentry{module}}\index{accnum\_to\_seqrecord() (en el módulo descarga)@\spxentry{accnum\_to\_seqrecord()}\spxextra{en el módulo descarga}}\phantomsection\label{\detokenize{descarga:module-descarga}}

\begin{fulllineitems}
\phantomsection\label{\detokenize{descarga:descarga.accnum_to_seqrecord}}
\pysigstartsignatures
\pysiglinewithargsret{\sphinxcode{\sphinxupquote{descarga.}}\sphinxbfcode{\sphinxupquote{accnum\_to\_seqrecord}}}{\sphinxparam{\DUrole{n}{accesion\_number}}}{}
\pysigstopsignatures
\sphinxAtStartPar
Función que recibe un accession number de la base de datos “nucleotide” de NCBI
\sphinxurl{https://www.ncbi.nlm.nih.gov/nuccore}.
Genera el archivo .gbk (GenBank) en la carpeta “files” y retorna la secuencia en formato SeqRecord.
\begin{quote}\begin{description}
\sphinxlineitem{Parámetros}
\sphinxAtStartPar
\sphinxstyleliteralstrong{\sphinxupquote{accesion\_number}} \textendash{} El accesion number del archivo en GenBank, por ejemplo NG\_008617.1

\sphinxlineitem{Devuelve}
\sphinxAtStartPar
La secuencia en formato SeqRecord con todas sus anotaciones.

\end{description}\end{quote}

\end{fulllineitems}

\index{main() (en el módulo descarga)@\spxentry{main()}\spxextra{en el módulo descarga}}

\begin{fulllineitems}
\phantomsection\label{\detokenize{descarga:descarga.main}}
\pysigstartsignatures
\pysiglinewithargsret{\sphinxcode{\sphinxupquote{descarga.}}\sphinxbfcode{\sphinxupquote{main}}}{}{}
\pysigstopsignatures
\sphinxAtStartPar
Ejecutar directamente el archivo “descarga” requiere entregar el parámetro \textendash{}accessionnumber para
descargar una secuencia mediante la función accnum\_to\_seqrecord.
\begin{quote}\begin{description}
\sphinxlineitem{Parámetros}
\sphinxAtStartPar
\sphinxstyleliteralstrong{\sphinxupquote{\sphinxhyphen{}\sphinxhyphen{}accesionnumber}} \textendash{} El accesion number del archivo en GenBank, por ejemplo NG\_008617.1

\end{description}\end{quote}

\end{fulllineitems}

\index{parse\_file\_to\_seqrecord() (en el módulo descarga)@\spxentry{parse\_file\_to\_seqrecord()}\spxextra{en el módulo descarga}}

\begin{fulllineitems}
\phantomsection\label{\detokenize{descarga:descarga.parse_file_to_seqrecord}}
\pysigstartsignatures
\pysiglinewithargsret{\sphinxcode{\sphinxupquote{descarga.}}\sphinxbfcode{\sphinxupquote{parse\_file\_to\_seqrecord}}}{\sphinxparam{\DUrole{n}{filepath}}}{}
\pysigstopsignatures
\sphinxAtStartPar
Función que recibe la ruta de un archivo genético anotado y lo retorna como SeqRecord. Genera una copia
del achivo en la carpeta files. Solo se aceptan archivos en formato «GenBank», o sea “.gbk” y
“.gb”. Estos archivos pueden estar comprimidos en formato “.gz”.
\begin{quote}\begin{description}
\sphinxlineitem{Parámetros}
\sphinxAtStartPar
\sphinxstyleliteralstrong{\sphinxupquote{filepath}} \textendash{} La ruta donde se encuentra el archivo que se quiere utilizar

\sphinxlineitem{Devuelve}
\sphinxAtStartPar
La secuencia en formato SeqRecord con todas sus anotaciones.

\end{description}\end{quote}

\end{fulllineitems}


\sphinxstepscope


\section{diseno module}
\label{\detokenize{diseno:diseno-module}}\label{\detokenize{diseno::doc}}
\sphinxAtStartPar
El módulo diseño contiene las funciones que permiten ejecutar el algoritmo diseñador y multiplexador de sondas. El diseño de las sondas de hibridación requiere de cierto procesamiento de datos para poder obtener el resultado adecuado, desde obtener las regiones de interés hasta verificar la validez de las sondas.

\sphinxAtStartPar
Este módulo está desarrollado para ser importado e utilizado por una interfaz de usuario.
\index{module@\spxentry{module}!diseno@\spxentry{diseno}}\index{diseno@\spxentry{diseno}!module@\spxentry{module}}\index{check\_probe() (en el módulo diseno)@\spxentry{check\_probe()}\spxextra{en el módulo diseno}}\phantomsection\label{\detokenize{diseno:module-diseno}}

\begin{fulllineitems}
\phantomsection\label{\detokenize{diseno:diseno.check_probe}}
\pysigstartsignatures
\pysiglinewithargsret{\sphinxcode{\sphinxupquote{diseno.}}\sphinxbfcode{\sphinxupquote{check\_probe}}}{\sphinxparam{\DUrole{n}{seq}}\sphinxparamcomma \sphinxparam{\DUrole{n}{iscentral}}\sphinxparamcomma \sphinxparam{\DUrole{n}{minlen}}\sphinxparamcomma \sphinxparam{\DUrole{n}{maxlen}}\sphinxparamcomma \sphinxparam{\DUrole{n}{tmmin}}\sphinxparamcomma \sphinxparam{\DUrole{n}{tmmax}}\sphinxparamcomma \sphinxparam{\DUrole{n}{gcmin}}\sphinxparamcomma \sphinxparam{\DUrole{n}{gcmax}}\sphinxparamcomma \sphinxparam{\DUrole{n}{dgmin\_homodim}}\sphinxparamcomma \sphinxparam{\DUrole{n}{dgmin\_hairpin}}\sphinxparamcomma \sphinxparam{\DUrole{n}{maxhomopol\_simple}}\sphinxparamcomma \sphinxparam{\DUrole{n}{maxhomopol\_double}}\sphinxparamcomma \sphinxparam{\DUrole{n}{maxhomopol\_triple}}}{}
\pysigstopsignatures
\sphinxAtStartPar
Función que recibe una sonda (secuencia de nucleótidos) y las restricciones para verificar que la
sonda cumpla los parámetros establecidos.
Retorna un valor booleano que determina si cumple las restricciones o no.
\begin{quote}\begin{description}
\sphinxlineitem{Parámetros}\begin{itemize}
\item {} 
\sphinxAtStartPar
\sphinxstyleliteralstrong{\sphinxupquote{seq}} \textendash{} La secuencia que se quiere verificar.

\item {} 
\sphinxAtStartPar
\sphinxstyleliteralstrong{\sphinxupquote{iscentral}} \textendash{} Booleano que indica si la sonda es central (una mitad en un exón y la otra en el siguiente)

\item {} 
\sphinxAtStartPar
\sphinxstyleliteralstrong{\sphinxupquote{minlen}} \textendash{} El largo mínimo de la sonda.

\item {} 
\sphinxAtStartPar
\sphinxstyleliteralstrong{\sphinxupquote{maxlen}} \textendash{} El largo máximo de la sonda.

\item {} 
\sphinxAtStartPar
\sphinxstyleliteralstrong{\sphinxupquote{tmmin}} \textendash{} Temperatura de melting mínima.

\item {} 
\sphinxAtStartPar
\sphinxstyleliteralstrong{\sphinxupquote{tmmax}} \textendash{} Temperatura de melting máxima.

\item {} 
\sphinxAtStartPar
\sphinxstyleliteralstrong{\sphinxupquote{gcmin}} \textendash{} Porcentaje de GC mínimo.

\item {} 
\sphinxAtStartPar
\sphinxstyleliteralstrong{\sphinxupquote{gcmax}} \textendash{} Porcentaje de GC máximo.

\item {} 
\sphinxAtStartPar
\sphinxstyleliteralstrong{\sphinxupquote{dgmin\_homodim}} \textendash{} Delta G mínimo permitido para validación de homodimerización.

\item {} 
\sphinxAtStartPar
\sphinxstyleliteralstrong{\sphinxupquote{dgmin\_hairpin}} \textendash{} Delta G mínimo permitido para validación de hairpin u horquilla.

\item {} 
\sphinxAtStartPar
\sphinxstyleliteralstrong{\sphinxupquote{maxhomopol\_simple}} \textendash{} Cantidad máxima permitida de homopolímeros simples (AAAAA)

\item {} 
\sphinxAtStartPar
\sphinxstyleliteralstrong{\sphinxupquote{maxhomopol\_double}} \textendash{} Cantidad máxima permitida de homopolímeros dobles (AGAGAGAG)

\item {} 
\sphinxAtStartPar
\sphinxstyleliteralstrong{\sphinxupquote{maxhomopol\_triple}} \textendash{} Cantidad máxima permitida de homopolímeros triples (AGCAGCAGC)

\end{itemize}

\sphinxlineitem{Devuelve}
\sphinxAtStartPar
Booleano que determina la validez de la secuencia

\end{description}\end{quote}

\end{fulllineitems}

\index{generate\_xlsx() (en el módulo diseno)@\spxentry{generate\_xlsx()}\spxextra{en el módulo diseno}}

\begin{fulllineitems}
\phantomsection\label{\detokenize{diseno:diseno.generate_xlsx}}
\pysigstartsignatures
\pysiglinewithargsret{\sphinxcode{\sphinxupquote{diseno.}}\sphinxbfcode{\sphinxupquote{generate\_xlsx}}}{\sphinxparam{\DUrole{n}{df}}\sphinxparamcomma \sphinxparam{\DUrole{n}{name}}\sphinxparamcomma \sphinxparam{\DUrole{n}{genes}}\sphinxparamcomma \sphinxparam{\DUrole{n}{minlen}\DUrole{o}{=}\DUrole{default_value}{60}}\sphinxparamcomma \sphinxparam{\DUrole{n}{maxlen}\DUrole{o}{=}\DUrole{default_value}{120}}\sphinxparamcomma \sphinxparam{\DUrole{n}{tmmin}\DUrole{o}{=}\DUrole{default_value}{65}}\sphinxparamcomma \sphinxparam{\DUrole{n}{tmmax}\DUrole{o}{=}\DUrole{default_value}{80}}\sphinxparamcomma \sphinxparam{\DUrole{n}{gcmin}\DUrole{o}{=}\DUrole{default_value}{30}}\sphinxparamcomma \sphinxparam{\DUrole{n}{gcmax}\DUrole{o}{=}\DUrole{default_value}{70}}\sphinxparamcomma \sphinxparam{\DUrole{n}{mindist}\DUrole{o}{=}\DUrole{default_value}{0}}\sphinxparamcomma \sphinxparam{\DUrole{n}{maxdist}\DUrole{o}{=}\DUrole{default_value}{50}}\sphinxparamcomma \sphinxparam{\DUrole{n}{minoverlap}\DUrole{o}{=}\DUrole{default_value}{25}}\sphinxparamcomma \sphinxparam{\DUrole{n}{maxoverlap}\DUrole{o}{=}\DUrole{default_value}{50}}\sphinxparamcomma \sphinxparam{\DUrole{n}{dgmin\_homodim}\DUrole{o}{=}\DUrole{default_value}{\sphinxhyphen{}10000}}\sphinxparamcomma \sphinxparam{\DUrole{n}{dgmin\_hairpin}\DUrole{o}{=}\DUrole{default_value}{\sphinxhyphen{}10000}}\sphinxparamcomma \sphinxparam{\DUrole{n}{maxhomopol\_simple}\DUrole{o}{=}\DUrole{default_value}{6}}\sphinxparamcomma \sphinxparam{\DUrole{n}{maxhomopol\_double}\DUrole{o}{=}\DUrole{default_value}{5}}\sphinxparamcomma \sphinxparam{\DUrole{n}{maxhomopol\_triple}\DUrole{o}{=}\DUrole{default_value}{4}}\sphinxparamcomma \sphinxparam{\DUrole{n}{multiplex}\DUrole{o}{=}\DUrole{default_value}{True}}\sphinxparamcomma \sphinxparam{\DUrole{n}{mindg}\DUrole{o}{=}\DUrole{default_value}{\sphinxhyphen{}13627}}\sphinxparamcomma \sphinxparam{\DUrole{n}{maxdt}\DUrole{o}{=}\DUrole{default_value}{5}}\sphinxparamcomma \sphinxparam{\DUrole{n}{tiempo\_diseno}\DUrole{o}{=}\DUrole{default_value}{0}}}{}
\pysigstopsignatures
\sphinxAtStartPar
Función que genera un reporte excel que contiene todas las sondas generadas y sus características. Además muestra las restricciones iniciales y los grupos de multiplexación.
\begin{quote}\begin{description}
\sphinxlineitem{Parámetros}\begin{itemize}
\item {} 
\sphinxAtStartPar
\sphinxstyleliteralstrong{\sphinxupquote{df}} \textendash{} DataFrame de pandas que contiene las sondas y sus parámetros.

\item {} 
\sphinxAtStartPar
\sphinxstyleliteralstrong{\sphinxupquote{name}} \textendash{} Nombre que se le quiere dar al reporte.

\item {} 
\sphinxAtStartPar
\sphinxstyleliteralstrong{\sphinxupquote{minlen}} \textendash{} El largo mínimo de la sonda.

\item {} 
\sphinxAtStartPar
\sphinxstyleliteralstrong{\sphinxupquote{maxlen}} \textendash{} El largo máximo de la sonda.

\item {} 
\sphinxAtStartPar
\sphinxstyleliteralstrong{\sphinxupquote{tmmin}} \textendash{} Temperatura de melting mínima.

\item {} 
\sphinxAtStartPar
\sphinxstyleliteralstrong{\sphinxupquote{tmmax}} \textendash{} Temperatura de melting máxima.

\item {} 
\sphinxAtStartPar
\sphinxstyleliteralstrong{\sphinxupquote{gcmin}} \textendash{} Porcentaje de GC mínimo.

\item {} 
\sphinxAtStartPar
\sphinxstyleliteralstrong{\sphinxupquote{gcmin}} \textendash{} Porcentaje de GC máximo.

\item {} 
\sphinxAtStartPar
\sphinxstyleliteralstrong{\sphinxupquote{mindist}} \textendash{} Distancia mínima al borde del exón.

\item {} 
\sphinxAtStartPar
\sphinxstyleliteralstrong{\sphinxupquote{maxdist}} \textendash{} Distancia máxima al borde del exón.

\item {} 
\sphinxAtStartPar
\sphinxstyleliteralstrong{\sphinxupquote{minoverlap}} \textendash{} Sobrelape mínimo entre sondas.

\item {} 
\sphinxAtStartPar
\sphinxstyleliteralstrong{\sphinxupquote{maxoverlap}} \textendash{} Sobrelape máximo entre sondas.

\item {} 
\sphinxAtStartPar
\sphinxstyleliteralstrong{\sphinxupquote{dgmin\_homodim}} \textendash{} Delta G mínimo permitido para validación de homodimerización.

\item {} 
\sphinxAtStartPar
\sphinxstyleliteralstrong{\sphinxupquote{dgmin\_hairpin}} \textendash{} Delta G mínimo permitido para validación de hairpin u horquilla.

\item {} 
\sphinxAtStartPar
\sphinxstyleliteralstrong{\sphinxupquote{maxhomopol\_simple}} \textendash{} Cantidad máxima permitida de homopolímeros simples (AAAAA)

\item {} 
\sphinxAtStartPar
\sphinxstyleliteralstrong{\sphinxupquote{maxhomopol\_double}} \textendash{} Cantidad máxima permitida de homopolímeros dobles (AGAGAGAG)

\item {} 
\sphinxAtStartPar
\sphinxstyleliteralstrong{\sphinxupquote{maxhomopol\_triple}} \textendash{} Cantidad máxima permitida de homopolímeros triples (AGCAGCAGC)

\item {} 
\sphinxAtStartPar
\sphinxstyleliteralstrong{\sphinxupquote{multiplex}} \textendash{} Booleano que indica si el diseño considera multiplexación

\item {} 
\sphinxAtStartPar
\sphinxstyleliteralstrong{\sphinxupquote{mindg}} \textendash{} Delta G mínimo para heterodimerización (multiplex)

\item {} 
\sphinxAtStartPar
\sphinxstyleliteralstrong{\sphinxupquote{maxdt}} \textendash{} Diferencia máxima de Tm (multiplex)

\item {} 
\sphinxAtStartPar
\sphinxstyleliteralstrong{\sphinxupquote{tiempo\_diseno}} \textendash{} Tiempo que demoró la ejecución en segundos

\end{itemize}

\sphinxlineitem{Devuelve}
\sphinxAtStartPar
Nombre del archivo XLSX generado

\end{description}\end{quote}

\end{fulllineitems}

\index{get\_all\_genes() (en el módulo diseno)@\spxentry{get\_all\_genes()}\spxextra{en el módulo diseno}}

\begin{fulllineitems}
\phantomsection\label{\detokenize{diseno:diseno.get_all_genes}}
\pysigstartsignatures
\pysiglinewithargsret{\sphinxcode{\sphinxupquote{diseno.}}\sphinxbfcode{\sphinxupquote{get\_all\_genes}}}{\sphinxparam{\DUrole{n}{seqrecord}}}{}
\pysigstopsignatures
\sphinxAtStartPar
Obtener todos los genes anotados
\begin{quote}\begin{description}
\sphinxlineitem{Parámetros}
\sphinxAtStartPar
\sphinxstyleliteralstrong{\sphinxupquote{seqrecord}} \textendash{} Subregiones que representan la secuencia codificante o secuencia de exones

\sphinxlineitem{Devuelve}
\sphinxAtStartPar
Lista con pares de posiciones en una tupla: el fin de un exón y el inicio del siguiente.

\end{description}\end{quote}

\end{fulllineitems}

\index{get\_all\_transcripts() (en el módulo diseno)@\spxentry{get\_all\_transcripts()}\spxextra{en el módulo diseno}}

\begin{fulllineitems}
\phantomsection\label{\detokenize{diseno:diseno.get_all_transcripts}}
\pysigstartsignatures
\pysiglinewithargsret{\sphinxcode{\sphinxupquote{diseno.}}\sphinxbfcode{\sphinxupquote{get\_all\_transcripts}}}{\sphinxparam{\DUrole{n}{seqrecord}}}{}
\pysigstopsignatures
\sphinxAtStartPar
Función que retorna todos los transcritos anotados dentro de una secuancia.
\begin{quote}\begin{description}
\sphinxlineitem{Parámetros}
\sphinxAtStartPar
\sphinxstyleliteralstrong{\sphinxupquote{seqrecord}} \textendash{} Subregiones que representan la secuencia codificante o secuencia de exones

\sphinxlineitem{Devuelve}
\sphinxAtStartPar
Lista con pares de posiciones en una tupla: el fin de un exón y el inicio del siguiente.

\end{description}\end{quote}

\end{fulllineitems}

\index{get\_probes\_from\_pos() (en el módulo diseno)@\spxentry{get\_probes\_from\_pos()}\spxextra{en el módulo diseno}}

\begin{fulllineitems}
\phantomsection\label{\detokenize{diseno:diseno.get_probes_from_pos}}
\pysigstartsignatures
\pysiglinewithargsret{\sphinxcode{\sphinxupquote{diseno.}}\sphinxbfcode{\sphinxupquote{get\_probes\_from\_pos}}}{\sphinxparam{\DUrole{n}{empalme}}\sphinxparamcomma \sphinxparam{\DUrole{n}{record}}\sphinxparamcomma \sphinxparam{\DUrole{n}{site}}\sphinxparamcomma \sphinxparam{\DUrole{n}{minlen}}\sphinxparamcomma \sphinxparam{\DUrole{n}{maxlen}}\sphinxparamcomma \sphinxparam{\DUrole{n}{tmmin}}\sphinxparamcomma \sphinxparam{\DUrole{n}{tmmax}}\sphinxparamcomma \sphinxparam{\DUrole{n}{gcmin}}\sphinxparamcomma \sphinxparam{\DUrole{n}{gcmax}}\sphinxparamcomma \sphinxparam{\DUrole{n}{mindist}}\sphinxparamcomma \sphinxparam{\DUrole{n}{maxdist}}\sphinxparamcomma \sphinxparam{\DUrole{n}{minoverlap}}\sphinxparamcomma \sphinxparam{\DUrole{n}{maxoverlap}}\sphinxparamcomma \sphinxparam{\DUrole{n}{dgmin\_homodim}}\sphinxparamcomma \sphinxparam{\DUrole{n}{dgmin\_hairpin}}\sphinxparamcomma \sphinxparam{\DUrole{n}{maxhomopol\_simple}}\sphinxparamcomma \sphinxparam{\DUrole{n}{maxhomopol\_double}}\sphinxparamcomma \sphinxparam{\DUrole{n}{maxhomopol\_triple}}}{}
\pysigstopsignatures
\sphinxAtStartPar
Esta función genera un par de sondas a partir de la secuencia de referencia y dos posiciones que representan el punto de empalme entre dos exones.
El par de sondas generadas pueden ubicarse en el exón donor, acceptor o en ambos (central).
\begin{quote}\begin{description}
\sphinxlineitem{Parámetros}\begin{itemize}
\item {} 
\sphinxAtStartPar
\sphinxstyleliteralstrong{\sphinxupquote{empalme}} \textendash{} Tupla con posiciones que representan al punto de empalme o splicing.

\item {} 
\sphinxAtStartPar
\sphinxstyleliteralstrong{\sphinxupquote{record}} \textendash{} Secuencia de referencia.

\item {} 
\sphinxAtStartPar
\sphinxstyleliteralstrong{\sphinxupquote{site}} \textendash{} Entero que representa si las sondas son donor, acceptor o central.

\item {} 
\sphinxAtStartPar
\sphinxstyleliteralstrong{\sphinxupquote{minlen}} \textendash{} El largo mínimo de la sonda.

\item {} 
\sphinxAtStartPar
\sphinxstyleliteralstrong{\sphinxupquote{maxlen}} \textendash{} El largo máximo de la sonda.

\item {} 
\sphinxAtStartPar
\sphinxstyleliteralstrong{\sphinxupquote{tmmin}} \textendash{} Temperatura de melting mínima.

\item {} 
\sphinxAtStartPar
\sphinxstyleliteralstrong{\sphinxupquote{tmmax}} \textendash{} Temperatura de melting máxima.

\item {} 
\sphinxAtStartPar
\sphinxstyleliteralstrong{\sphinxupquote{gcmin}} \textendash{} Porcentaje de GC mínimo.

\item {} 
\sphinxAtStartPar
\sphinxstyleliteralstrong{\sphinxupquote{gcmin}} \textendash{} Porcentaje de GC máximo.

\item {} 
\sphinxAtStartPar
\sphinxstyleliteralstrong{\sphinxupquote{mindist}} \textendash{} Distancia mínima al borde del exón.

\item {} 
\sphinxAtStartPar
\sphinxstyleliteralstrong{\sphinxupquote{maxdist}} \textendash{} Distancia máxima al borde del exón.

\item {} 
\sphinxAtStartPar
\sphinxstyleliteralstrong{\sphinxupquote{minoverlap}} \textendash{} Sobrelape mínimo entre sondas.

\item {} 
\sphinxAtStartPar
\sphinxstyleliteralstrong{\sphinxupquote{maxoverlap}} \textendash{} Sobrelape máximo entre sondas.

\item {} 
\sphinxAtStartPar
\sphinxstyleliteralstrong{\sphinxupquote{dgmin\_homodim}} \textendash{} Delta G mínimo permitido para validación de homodimerización.

\item {} 
\sphinxAtStartPar
\sphinxstyleliteralstrong{\sphinxupquote{dgmin\_hairpin}} \textendash{} Delta G mínimo permitido para validación de hairpin u horquilla.

\item {} 
\sphinxAtStartPar
\sphinxstyleliteralstrong{\sphinxupquote{maxhomopol\_simple}} \textendash{} Cantidad máxima permitida de homopolímeros simples (AAAAA)

\item {} 
\sphinxAtStartPar
\sphinxstyleliteralstrong{\sphinxupquote{maxhomopol\_double}} \textendash{} Cantidad máxima permitida de homopolímeros dobles (AGAGAGAG)

\item {} 
\sphinxAtStartPar
\sphinxstyleliteralstrong{\sphinxupquote{maxhomopol\_triple}} \textendash{} Cantidad máxima permitida de homopolímeros triples (AGCAGCAGC)

\end{itemize}

\sphinxlineitem{Devuelve}
\sphinxAtStartPar
Lista con dos sondas, cada sonda es una tupla con la secuencia, la posición de inicio, la posición final y la distancia al borde del exón si aplica.

\end{description}\end{quote}

\end{fulllineitems}

\index{get\_splicing\_pairs() (en el módulo diseno)@\spxentry{get\_splicing\_pairs()}\spxextra{en el módulo diseno}}

\begin{fulllineitems}
\phantomsection\label{\detokenize{diseno:diseno.get_splicing_pairs}}
\pysigstartsignatures
\pysiglinewithargsret{\sphinxcode{\sphinxupquote{diseno.}}\sphinxbfcode{\sphinxupquote{get\_splicing\_pairs}}}{\sphinxparam{\DUrole{n}{locations}}}{}
\pysigstopsignatures
\sphinxAtStartPar
Función que obtiene las zonas de empalme a partir de una serie de regiones que representan una transcripción.
Las zonas de empalme se representan por dos posiciones: el fin de un exón y el inicio del siguiente.
\begin{quote}\begin{description}
\sphinxlineitem{Parámetros}
\sphinxAtStartPar
\sphinxstyleliteralstrong{\sphinxupquote{locations}} \textendash{} Subregiones que representan la secuencia codificante o secuencia de exones

\sphinxlineitem{Devuelve}
\sphinxAtStartPar
Lista con pares de posiciones en una tupla: el fin de un exón y el inicio del siguiente.

\end{description}\end{quote}

\end{fulllineitems}

\index{get\_splicings() (en el módulo diseno)@\spxentry{get\_splicings()}\spxextra{en el módulo diseno}}

\begin{fulllineitems}
\phantomsection\label{\detokenize{diseno:diseno.get_splicings}}
\pysigstartsignatures
\pysiglinewithargsret{\sphinxcode{\sphinxupquote{diseno.}}\sphinxbfcode{\sphinxupquote{get\_splicings}}}{\sphinxparam{\DUrole{n}{seqrecord}}\sphinxparamcomma \sphinxparam{\DUrole{n}{transcripts}}}{}
\pysigstopsignatures
\sphinxAtStartPar
Función que retorna los empalmes sin repetición de una secuencia anotada en ciertos transcritos.
\begin{quote}\begin{description}
\sphinxlineitem{Parámetros}\begin{itemize}
\item {} 
\sphinxAtStartPar
\sphinxstyleliteralstrong{\sphinxupquote{seqrecord}} \textendash{} Secuencia anotada de referencia

\item {} 
\sphinxAtStartPar
\sphinxstyleliteralstrong{\sphinxupquote{transcripts}} \textendash{} Lista de transcritos

\end{itemize}

\sphinxlineitem{Devuelve}
\sphinxAtStartPar
Lista con pares de posiciones en una tupla: el fin de un exón y el inicio del siguiente.

\end{description}\end{quote}

\end{fulllineitems}

\index{main() (en el módulo diseno)@\spxentry{main()}\spxextra{en el módulo diseno}}

\begin{fulllineitems}
\phantomsection\label{\detokenize{diseno:diseno.main}}
\pysigstartsignatures
\pysiglinewithargsret{\sphinxcode{\sphinxupquote{diseno.}}\sphinxbfcode{\sphinxupquote{main}}}{}{}
\pysigstopsignatures
\end{fulllineitems}

\index{probe\_designer() (en el módulo diseno)@\spxentry{probe\_designer()}\spxextra{en el módulo diseno}}

\begin{fulllineitems}
\phantomsection\label{\detokenize{diseno:diseno.probe_designer}}
\pysigstartsignatures
\pysiglinewithargsret{\sphinxcode{\sphinxupquote{diseno.}}\sphinxbfcode{\sphinxupquote{probe\_designer}}}{\sphinxparam{\DUrole{n}{record}}\sphinxparamcomma \sphinxparam{\DUrole{n}{transcripts}}\sphinxparamcomma \sphinxparam{\DUrole{n}{progreso\_queue}}\sphinxparamcomma \sphinxparam{\DUrole{n}{minlen}\DUrole{o}{=}\DUrole{default_value}{60}}\sphinxparamcomma \sphinxparam{\DUrole{n}{maxlen}\DUrole{o}{=}\DUrole{default_value}{120}}\sphinxparamcomma \sphinxparam{\DUrole{n}{tmmin}\DUrole{o}{=}\DUrole{default_value}{65}}\sphinxparamcomma \sphinxparam{\DUrole{n}{tmmax}\DUrole{o}{=}\DUrole{default_value}{80}}\sphinxparamcomma \sphinxparam{\DUrole{n}{gcmin}\DUrole{o}{=}\DUrole{default_value}{30}}\sphinxparamcomma \sphinxparam{\DUrole{n}{gcmax}\DUrole{o}{=}\DUrole{default_value}{70}}\sphinxparamcomma \sphinxparam{\DUrole{n}{mindist}\DUrole{o}{=}\DUrole{default_value}{0}}\sphinxparamcomma \sphinxparam{\DUrole{n}{maxdist}\DUrole{o}{=}\DUrole{default_value}{200}}\sphinxparamcomma \sphinxparam{\DUrole{n}{minoverlap}\DUrole{o}{=}\DUrole{default_value}{25}}\sphinxparamcomma \sphinxparam{\DUrole{n}{maxoverlap}\DUrole{o}{=}\DUrole{default_value}{50}}\sphinxparamcomma \sphinxparam{\DUrole{n}{dgmin\_homodim}\DUrole{o}{=}\DUrole{default_value}{\sphinxhyphen{}10000}}\sphinxparamcomma \sphinxparam{\DUrole{n}{dgmin\_hairpin}\DUrole{o}{=}\DUrole{default_value}{\sphinxhyphen{}10000}}\sphinxparamcomma \sphinxparam{\DUrole{n}{maxhomopol\_simple}\DUrole{o}{=}\DUrole{default_value}{6}}\sphinxparamcomma \sphinxparam{\DUrole{n}{maxhomopol\_double}\DUrole{o}{=}\DUrole{default_value}{5}}\sphinxparamcomma \sphinxparam{\DUrole{n}{maxhomopol\_triple}\DUrole{o}{=}\DUrole{default_value}{4}}\sphinxparamcomma \sphinxparam{\DUrole{n}{multiplex}\DUrole{o}{=}\DUrole{default_value}{True}}\sphinxparamcomma \sphinxparam{\DUrole{n}{mindg}\DUrole{o}{=}\DUrole{default_value}{\sphinxhyphen{}13627}}\sphinxparamcomma \sphinxparam{\DUrole{n}{maxdt}\DUrole{o}{=}\DUrole{default_value}{5}}}{}
\pysigstopsignatures
\sphinxAtStartPar
Función principal diseñadora de sondas. De la secuencia anotada se obtiene una serie de sondas como subsecuencias de ésta, que cumplen varias restricciones ajustadas por los parámetros.
\begin{quote}\begin{description}
\sphinxlineitem{Parámetros}\begin{itemize}
\item {} 
\sphinxAtStartPar
\sphinxstyleliteralstrong{\sphinxupquote{record}} \textendash{} Secuencia anotada de referencia para la creación de las sondas.

\item {} 
\sphinxAtStartPar
\sphinxstyleliteralstrong{\sphinxupquote{transcripts}} \textendash{} Transcripciones en las cuales se quiere diseñar.

\item {} 
\sphinxAtStartPar
\sphinxstyleliteralstrong{\sphinxupquote{minlen}} \textendash{} El largo mínimo de la sonda.

\item {} 
\sphinxAtStartPar
\sphinxstyleliteralstrong{\sphinxupquote{maxlen}} \textendash{} El largo máximo de la sonda.

\item {} 
\sphinxAtStartPar
\sphinxstyleliteralstrong{\sphinxupquote{tmmin}} \textendash{} Temperatura de melting mínima.

\item {} 
\sphinxAtStartPar
\sphinxstyleliteralstrong{\sphinxupquote{tmmax}} \textendash{} Temperatura de melting máxima.

\item {} 
\sphinxAtStartPar
\sphinxstyleliteralstrong{\sphinxupquote{gcmin}} \textendash{} Porcentaje de GC mínimo.

\item {} 
\sphinxAtStartPar
\sphinxstyleliteralstrong{\sphinxupquote{gcmin}} \textendash{} Porcentaje de GC máximo.

\item {} 
\sphinxAtStartPar
\sphinxstyleliteralstrong{\sphinxupquote{mindist}} \textendash{} Distancia mínima al borde del exón.

\item {} 
\sphinxAtStartPar
\sphinxstyleliteralstrong{\sphinxupquote{maxdist}} \textendash{} Distancia máxima al borde del exón.

\item {} 
\sphinxAtStartPar
\sphinxstyleliteralstrong{\sphinxupquote{minoverlap}} \textendash{} Sobrelape mínimo entre sondas.

\item {} 
\sphinxAtStartPar
\sphinxstyleliteralstrong{\sphinxupquote{maxoverlap}} \textendash{} Sobrelape máximo entre sondas.

\item {} 
\sphinxAtStartPar
\sphinxstyleliteralstrong{\sphinxupquote{dgmin\_homodim}} \textendash{} Delta G mínimo permitido para validación de homodimerización.

\item {} 
\sphinxAtStartPar
\sphinxstyleliteralstrong{\sphinxupquote{dgmin\_hairpin}} \textendash{} Delta G mínimo permitido para validación de hairpin u horquilla.

\item {} 
\sphinxAtStartPar
\sphinxstyleliteralstrong{\sphinxupquote{maxhomopol\_simple}} \textendash{} Cantidad máxima permitida de homopolímeros simples (AAAAA)

\item {} 
\sphinxAtStartPar
\sphinxstyleliteralstrong{\sphinxupquote{maxhomopol\_double}} \textendash{} Cantidad máxima permitida de homopolímeros dobles (AGAGAGAG)

\item {} 
\sphinxAtStartPar
\sphinxstyleliteralstrong{\sphinxupquote{maxhomopol\_triple}} \textendash{} Cantidad máxima permitida de homopolímeros triples (AGCAGCAGC)

\item {} 
\sphinxAtStartPar
\sphinxstyleliteralstrong{\sphinxupquote{multiplex}} \textendash{} Booleano que indica si el diseño considera multiplexación

\item {} 
\sphinxAtStartPar
\sphinxstyleliteralstrong{\sphinxupquote{mindg}} \textendash{} Delta G mínimo para heterodimerización (multiplex)

\item {} 
\sphinxAtStartPar
\sphinxstyleliteralstrong{\sphinxupquote{maxdt}} \textendash{} Diferencia máxima de Tm (multiplex)

\end{itemize}

\sphinxlineitem{Devuelve}
\sphinxAtStartPar
Dataframe con las sondas y sus características

\end{description}\end{quote}

\end{fulllineitems}

\index{segundos\_a\_hms() (en el módulo diseno)@\spxentry{segundos\_a\_hms()}\spxextra{en el módulo diseno}}

\begin{fulllineitems}
\phantomsection\label{\detokenize{diseno:diseno.segundos_a_hms}}
\pysigstartsignatures
\pysiglinewithargsret{\sphinxcode{\sphinxupquote{diseno.}}\sphinxbfcode{\sphinxupquote{segundos\_a\_hms}}}{\sphinxparam{\DUrole{n}{segundos}}}{}
\pysigstopsignatures
\end{fulllineitems}

\index{verify\_specificity() (en el módulo diseno)@\spxentry{verify\_specificity()}\spxextra{en el módulo diseno}}

\begin{fulllineitems}
\phantomsection\label{\detokenize{diseno:diseno.verify_specificity}}
\pysigstartsignatures
\pysiglinewithargsret{\sphinxcode{\sphinxupquote{diseno.}}\sphinxbfcode{\sphinxupquote{verify\_specificity}}}{\sphinxparam{\DUrole{n}{seq}}\sphinxparamcomma \sphinxparam{\DUrole{n}{iscentral}}}{}
\pysigstopsignatures
\sphinxAtStartPar
Función que verifica la especificidad o unicidad de una secuencia usando la herramienta de alineamiento BLAST de manera local.
Es necesario tener instalado Blast+ y las bases de datos en la carpeta “refseq”.
\begin{quote}\begin{description}
\sphinxlineitem{Parámetros}\begin{itemize}
\item {} 
\sphinxAtStartPar
\sphinxstyleliteralstrong{\sphinxupquote{seq}} \textendash{} Secuencia de nuecleótidos que se quiere validar.

\item {} 
\sphinxAtStartPar
\sphinxstyleliteralstrong{\sphinxupquote{iscentral}} \textendash{} Booleano que indica si se trata de una sonda central. Se utilizan diferentes criterios para chequear la especificidad de una sonda central.

\end{itemize}

\sphinxlineitem{Devuelve}
\sphinxAtStartPar
Booleano que indica si la secuencia se considera específica.

\end{description}\end{quote}

\end{fulllineitems}


\sphinxstepscope


\section{multiplex module}
\label{\detokenize{multiplex:multiplex-module}}\label{\detokenize{multiplex::doc}}
\sphinxAtStartPar
El módulo multiplex contiene las funciones que utiliza el módulo diseñador para poder llevar a cabo la multiplexación de las sondas diseñadas.
\index{module@\spxentry{module}!multiplex@\spxentry{multiplex}}\index{multiplex@\spxentry{multiplex}!module@\spxentry{module}}\index{are\_sequences\_compatible() (en el módulo multiplex)@\spxentry{are\_sequences\_compatible()}\spxextra{en el módulo multiplex}}\phantomsection\label{\detokenize{multiplex:module-multiplex}}

\begin{fulllineitems}
\phantomsection\label{\detokenize{multiplex:multiplex.are_sequences_compatible}}
\pysigstartsignatures
\pysiglinewithargsret{\sphinxcode{\sphinxupquote{multiplex.}}\sphinxbfcode{\sphinxupquote{are\_sequences\_compatible}}}{\sphinxparam{\DUrole{n}{seq1}}\sphinxparamcomma \sphinxparam{\DUrole{n}{seq2}}\sphinxparamcomma \sphinxparam{\DUrole{n}{mindg}}\sphinxparamcomma \sphinxparam{\DUrole{n}{maxdt}}}{}
\pysigstopsignatures
\sphinxAtStartPar
Esta función determina si dos secuencias de ADN son compatibles. 
Debes personalizar las condiciones de compatibilidad según tus necesidades.
\begin{quote}\begin{description}
\sphinxlineitem{Parámetros}\begin{itemize}
\item {} 
\sphinxAtStartPar
\sphinxstyleliteralstrong{\sphinxupquote{seq1}} \textendash{} Primera secuencia de ADN.

\item {} 
\sphinxAtStartPar
\sphinxstyleliteralstrong{\sphinxupquote{seq2}} \textendash{} Segunda secuencia de ADN.

\end{itemize}

\sphinxlineitem{Devuelve}
\sphinxAtStartPar
True si son compatibles, False en caso contrario.

\end{description}\end{quote}

\end{fulllineitems}

\index{multiplex\_sequences() (en el módulo multiplex)@\spxentry{multiplex\_sequences()}\spxextra{en el módulo multiplex}}

\begin{fulllineitems}
\phantomsection\label{\detokenize{multiplex:multiplex.multiplex_sequences}}
\pysigstartsignatures
\pysiglinewithargsret{\sphinxcode{\sphinxupquote{multiplex.}}\sphinxbfcode{\sphinxupquote{multiplex\_sequences}}}{\sphinxparam{\DUrole{n}{sequences}}\sphinxparamcomma \sphinxparam{\DUrole{n}{progress\_queue}}\sphinxparamcomma \sphinxparam{\DUrole{n}{mindg}\DUrole{o}{=}\DUrole{default_value}{\sphinxhyphen{}13627}}\sphinxparamcomma \sphinxparam{\DUrole{n}{maxdt}\DUrole{o}{=}\DUrole{default_value}{5}}}{}
\pysigstopsignatures
\sphinxAtStartPar
Función principal de multiplexación de sondas. Recibe las secuencias, los parámetros de multiplexación y la cola para informar progreso.
\begin{quote}\begin{description}
\sphinxlineitem{Parámetros}\begin{itemize}
\item {} 
\sphinxAtStartPar
\sphinxstyleliteralstrong{\sphinxupquote{sequences}} \textendash{} Lista con todas las diferentes secuencias.

\item {} 
\sphinxAtStartPar
\sphinxstyleliteralstrong{\sphinxupquote{progress\_queue}} \textendash{} Cola para informar el progreso a la interfaz.

\item {} 
\sphinxAtStartPar
\sphinxstyleliteralstrong{\sphinxupquote{mindg}} \textendash{} Mínimo delta G heterodimerización para que dos secuencias sean compatibles.

\item {} 
\sphinxAtStartPar
\sphinxstyleliteralstrong{\sphinxupquote{maxdt}} \textendash{} Máxima diferencia de Tm para que dos secuencias sean compatibles.

\end{itemize}

\sphinxlineitem{Devuelve}
\sphinxAtStartPar
True si son compatibles, False en caso contrario.

\end{description}\end{quote}

\end{fulllineitems}



\chapter{Índices y tablas}
\label{\detokenize{index:indices-y-tablas}}\begin{itemize}
\item {} 
\sphinxAtStartPar
\DUrole{xref,std,std-ref}{genindex}

\item {} 
\sphinxAtStartPar
\DUrole{xref,std,std-ref}{modindex}

\item {} 
\sphinxAtStartPar
\DUrole{xref,std,std-ref}{search}

\end{itemize}


\renewcommand{\indexname}{Índice de Módulos Python}
\begin{sphinxtheindex}
\let\bigletter\sphinxstyleindexlettergroup
\bigletter{d}
\item\relax\sphinxstyleindexentry{descarga}\sphinxstyleindexpageref{descarga:\detokenize{module-descarga}}
\item\relax\sphinxstyleindexentry{diseno}\sphinxstyleindexpageref{diseno:\detokenize{module-diseno}}
\indexspace
\bigletter{m}
\item\relax\sphinxstyleindexentry{multiplex}\sphinxstyleindexpageref{multiplex:\detokenize{module-multiplex}}
\end{sphinxtheindex}

\renewcommand{\indexname}{Índice}
\printindex
\end{document}