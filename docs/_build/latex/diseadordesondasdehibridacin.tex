%% Generated by Sphinx.
\def\sphinxdocclass{report}
\documentclass[letterpaper,10pt,spanish]{sphinxmanual}
\ifdefined\pdfpxdimen
   \let\sphinxpxdimen\pdfpxdimen\else\newdimen\sphinxpxdimen
\fi \sphinxpxdimen=.75bp\relax
\ifdefined\pdfimageresolution
    \pdfimageresolution= \numexpr \dimexpr1in\relax/\sphinxpxdimen\relax
\fi
%% let collapsible pdf bookmarks panel have high depth per default
\PassOptionsToPackage{bookmarksdepth=5}{hyperref}

\PassOptionsToPackage{booktabs}{sphinx}
\PassOptionsToPackage{colorrows}{sphinx}

\PassOptionsToPackage{warn}{textcomp}
\usepackage[utf8]{inputenc}
\ifdefined\DeclareUnicodeCharacter
% support both utf8 and utf8x syntaxes
  \ifdefined\DeclareUnicodeCharacterAsOptional
    \def\sphinxDUC#1{\DeclareUnicodeCharacter{"#1}}
  \else
    \let\sphinxDUC\DeclareUnicodeCharacter
  \fi
  \sphinxDUC{00A0}{\nobreakspace}
  \sphinxDUC{2500}{\sphinxunichar{2500}}
  \sphinxDUC{2502}{\sphinxunichar{2502}}
  \sphinxDUC{2514}{\sphinxunichar{2514}}
  \sphinxDUC{251C}{\sphinxunichar{251C}}
  \sphinxDUC{2572}{\textbackslash}
\fi
\usepackage{cmap}
\usepackage[T1]{fontenc}
\usepackage{amsmath,amssymb,amstext}
\usepackage{babel}



\usepackage{tgtermes}
\usepackage{tgheros}
\renewcommand{\ttdefault}{txtt}



\usepackage[Sonny]{fncychap}
\ChNameVar{\Large\normalfont\sffamily}
\ChTitleVar{\Large\normalfont\sffamily}
\usepackage{sphinx}

\fvset{fontsize=auto}
\usepackage{geometry}


% Include hyperref last.
\usepackage{hyperref}
% Fix anchor placement for figures with captions.
\usepackage{hypcap}% it must be loaded after hyperref.
% Set up styles of URL: it should be placed after hyperref.
\urlstyle{same}

\addto\captionsspanish{\renewcommand{\contentsname}{Contents:}}

\usepackage{sphinxmessages}
\setcounter{tocdepth}{1}



\title{Diseñador de sondas de hibridación}
\date{24 de mayo de 2023}
\release{}
\author{Simón Vergara}
\newcommand{\sphinxlogo}{\vbox{}}
\renewcommand{\releasename}{}
\makeindex
\begin{document}

\ifdefined\shorthandoff
  \ifnum\catcode`\=\string=\active\shorthandoff{=}\fi
  \ifnum\catcode`\"=\active\shorthandoff{"}\fi
\fi

\pagestyle{empty}
\sphinxmaketitle
\pagestyle{plain}
\sphinxtableofcontents
\pagestyle{normal}
\phantomsection\label{\detokenize{index::doc}}


\sphinxstepscope


\chapter{Probe\sphinxhyphen{}Designer}
\label{\detokenize{modules:probe-designer}}\label{\detokenize{modules::doc}}
\sphinxstepscope


\section{descarga module}
\label{\detokenize{descarga:descarga-module}}\label{\detokenize{descarga::doc}}
\sphinxAtStartPar
Para obtener el ADN que se utilizará como referencia para el diseño de las sondas se va a descargar directamente el archivo genético desde la base de datos ‘Nucleotide’ de NCBI (National Center for Biotechnology Information) del NIH (National Institute of Health, EEUU) (\sphinxurl{https://www.ncbi.nlm.nih.gov/nuccore}). Esta base de datos contiene millones de registros de secuencias de nucleótidos con mucha información adicional, como la especie, cromosoma, nombre del gen, exones, la fuente de los datos, etc.

\sphinxAtStartPar
Para llevar a cabo la descarga de los archivos genéticos, se implementó un módulo que permite realizar descargas desde esta base de datos a través de la librería Entrez de BioPython y almacenar los archivos en una carpeta. De esta manera se puede automatizar la obtención de las secuencias de referencia.

\sphinxAtStartPar
Por otra parte se implementó la función que retorna una secuencia a partir de un archivo almacenado localmente, además de crear una copia del archivo y almacenarla. Esta función sólo podrá recibir archivos de secuencia anotados como GenBank y GFF3.

\sphinxAtStartPar
El producto de estas funciones implementadas es la secuencia en formato SeqRecord, una estructura de datos propia de la librería BioPython que permite almacenar grandes secuencias con su respectiva notación, como por ejemplo las posiciones de cada uno de los exones, las transcripciones, secciones no codificantes, entre otros. Además contiene información extra de la secuencia tal como la especie, el cromosoma, la fuente de donde salieron los datos, etc.

\phantomsection\label{\detokenize{descarga:module-descarga}}\index{módulo@\spxentry{módulo}!descarga@\spxentry{descarga}}\index{descarga@\spxentry{descarga}!módulo@\spxentry{módulo}}\index{accnum\_to\_seqrecord() (en el módulo descarga)@\spxentry{accnum\_to\_seqrecord()}\spxextra{en el módulo descarga}}

\begin{fulllineitems}
\phantomsection\label{\detokenize{descarga:descarga.accnum_to_seqrecord}}
\pysigstartsignatures
\pysiglinewithargsret{\sphinxcode{\sphinxupquote{descarga.}}\sphinxbfcode{\sphinxupquote{accnum\_to\_seqrecord}}}{\sphinxparam{\DUrole{n,n}{accesion\_number}}}{}
\pysigstopsignatures
\sphinxAtStartPar
Función que recibe un accession number de la base de datos “nucleotide” de NCBI
\sphinxurl{https://www.ncbi.nlm.nih.gov/nuccore}.
Genera el archivo .gbk (GenBank) en la carpeta “files” y retorna la secuencia en formato SeqRecord.
\begin{quote}\begin{description}
\sphinxlineitem{Parámetros}
\sphinxAtStartPar
\sphinxstyleliteralstrong{\sphinxupquote{accesion\_number}} \textendash{} El accesion number del archivo en GenBank, por ejemplo NG\_008617.1

\sphinxlineitem{Devuelve}
\sphinxAtStartPar
La secuencia en formato SeqRecord con todas sus anotaciones.

\end{description}\end{quote}

\end{fulllineitems}

\index{main() (en el módulo descarga)@\spxentry{main()}\spxextra{en el módulo descarga}}

\begin{fulllineitems}
\phantomsection\label{\detokenize{descarga:descarga.main}}
\pysigstartsignatures
\pysiglinewithargsret{\sphinxcode{\sphinxupquote{descarga.}}\sphinxbfcode{\sphinxupquote{main}}}{}{}
\pysigstopsignatures
\sphinxAtStartPar
Ejecutar directamente el archivo “descarga” requiere entregar el parámetro \textendash{}accessionnumber para
descargar una secuencia mediante la función accnum\_to\_seqrecord.
\begin{quote}\begin{description}
\sphinxlineitem{Parámetros}
\sphinxAtStartPar
\sphinxstyleliteralstrong{\sphinxupquote{\sphinxhyphen{}\sphinxhyphen{}accesionnumber}} \textendash{} El accesion number del archivo en GenBank, por ejemplo NG\_008617.1

\end{description}\end{quote}

\end{fulllineitems}

\index{parse\_file\_to\_seqrecord() (en el módulo descarga)@\spxentry{parse\_file\_to\_seqrecord()}\spxextra{en el módulo descarga}}

\begin{fulllineitems}
\phantomsection\label{\detokenize{descarga:descarga.parse_file_to_seqrecord}}
\pysigstartsignatures
\pysiglinewithargsret{\sphinxcode{\sphinxupquote{descarga.}}\sphinxbfcode{\sphinxupquote{parse\_file\_to\_seqrecord}}}{\sphinxparam{\DUrole{n,n}{filepath}}}{}
\pysigstopsignatures
\sphinxAtStartPar
Función que recibe la ruta de un archivo genético anotado y lo retorna como SeqRecord. Genera una copia
del achivo en la carpeta files. Solo se aceptan archivos en formato «GenBank» y «GFF3», o sea “.gbk”,
“.gb” y “.gff3”. Estos archivos pueden estar comprimidos en formato “.gz”.
\begin{quote}\begin{description}
\sphinxlineitem{Parámetros}
\sphinxAtStartPar
\sphinxstyleliteralstrong{\sphinxupquote{filepath}} \textendash{} La ruta donde se encuentra el archivo que se quiere utilizar

\sphinxlineitem{Devuelve}
\sphinxAtStartPar
La secuencia en formato SeqRecord con todas sus anotaciones.

\end{description}\end{quote}

\end{fulllineitems}


\sphinxstepscope


\section{test module}
\label{\detokenize{test:test-module}}\label{\detokenize{test::doc}}
\sphinxAtStartPar
Este módulo prueba las funciones implementadas en el módulo descarga, usando los métodos para obtener secuencias: mediante un archivo y mediante un accesion number. Accede a las ubicaciones de las anotaciones de las secuencias.

\phantomsection\label{\detokenize{test:module-test}}\index{módulo@\spxentry{módulo}!test@\spxentry{test}}\index{test@\spxentry{test}!módulo@\spxentry{módulo}}

\chapter{Índices y tablas}
\label{\detokenize{index:indices-y-tablas}}\begin{itemize}
\item {} 
\sphinxAtStartPar
\DUrole{xref,std,std-ref}{genindex}

\item {} 
\sphinxAtStartPar
\DUrole{xref,std,std-ref}{modindex}

\item {} 
\sphinxAtStartPar
\DUrole{xref,std,std-ref}{search}

\end{itemize}


\renewcommand{\indexname}{Índice de Módulos Python}
\begin{sphinxtheindex}
\let\bigletter\sphinxstyleindexlettergroup
\bigletter{d}
\item\relax\sphinxstyleindexentry{descarga}\sphinxstyleindexpageref{descarga:\detokenize{module-descarga}}
\indexspace
\bigletter{t}
\item\relax\sphinxstyleindexentry{test}\sphinxstyleindexpageref{test:\detokenize{module-test}}
\end{sphinxtheindex}

\renewcommand{\indexname}{Índice}
\printindex
\end{document}